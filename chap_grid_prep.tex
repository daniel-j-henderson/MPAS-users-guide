\chapter{Preparing Meshes}
\label{chap:mpas_grid_preparation}

This chapter describes the steps used to prepare SCVT meshes for use in MPAS-A.
For quasi-uniform meshes, very little preparation is actually needed, and
generally, one only needs to prepare mesh decomposition files --- files that
describe the decomposition of the SCVT mesh across processors --- when running
MPAS-A using multiple MPI tasks. The procedure for creating these mesh
decomposition files is described in the first section. 

For variable-resolution SCVT meshes, the area of mesh refinement may be rotated
to any part of the sphere using a program, grid\_rotate, described in the second
section. This utility program may be obtained from the MPAS-A download page.
\section{Graph partitioning with METIS} 
\label{sec:metis}

Before MPAS can be run in parallel, a mesh decomposition file with an
appropriate number of partitions (equal to the number of MPI tasks that will be
used) is required. A limited number of mesh decomposition files, named {\tt
graph.info.part.*}, are provided with each mesh, as is the mesh
connectivity file, named {\tt graph.info}. If the number of MPI tasks to be used when
running MPAS matches one of the pre-computed decomposition files, then there
is no need to run METIS.

In order to create new mesh decomposition files for some particular number of MPI tasks, 
only the {\tt graph.info} file is required.  The currently supported method for partitioning
a graph.info file uses the METIS software
(\url{http://glaros.dtc.umn.edu/gkhome/views/metis}).  The serial graph partitioning
program, METIS (rather than ParMETIS or hMETIS) should be sufficient for
quickly partitioning any mesh usable by MPAS.

After installing METIS, a {\tt graph.info} file may be partitioned into $N$
partitions by running

\vspace{12pt}
{\tt > gpmetis graph.info} $N$
\vspace{12pt}

\noindent where $N$ is the required number of partitions. The resulting file, {\tt graph.info.part.}$N$, 
can then be copied into the MPAS run directory before running the model with $N$ MPI tasks.


\section{Relocating refinement regions on the sphere}
\label{sec:grid_rotate} 

The purpose of the grid\_rotate program is simply to rotate an MPAS mesh file,
moving a refinement region from one geographic location to another, so that the
mesh can be re-used for different applications. This utility was developed out
of the need to save computational resources, since generating an SCVT ---
particularly one with a large number of generating points or a high degree of
refinement --- can take considerable time.

To build the grid\_rotate program, the Makefile should first be edited to set
the Fortran compiler to be used; if the NetCDF installation pointed to by the
{\tt NETCDF} environment variable was build with a separate Fortran interface
library, it will also be necessary to add {\tt -lnetcdff} just before {\tt -lnetcdf} in 
the Makefile. After editing the Makefile, running `make' should
result in a grid\_rotate executable file.

Besides the MPAS grid file to be rotated, grid\_rotate requires a namelist file,
{\tt namelist.input}, which specifies the rotation to be applied to the mesh.
The namelist variables are summarized in the table below
   
\vspace{12pt}
\begin{longtable}{|p{3.25in} |p{2.5in}|}
\hline
config\_original\_latitude\_degrees & original latitude of any point on the sphere \\ \hline
config\_original\_longitude\_degrees & original longitude of any point on the sphere \\ \hline
config\_new\_latitude\_degrees &  latitude to which the original point should be shifted \\ \hline
config\_new\_longitude\_degrees &  longitude to which the original point should be shifted \\ \hline
config\_birdseye\_rotation\_counter\_clockwise\_degrees & rotation about a vector from the sphere center through the original point \\ \hline
\end{longtable}
\vspace{12pt}

\noindent Essentially, one chooses any point on the sphere, decides where that
point should be shifted to, and specifies any change to the orientation (i.e.,
rotation) of the mesh about that point. 

Having set the rotation parameters in the {\tt namelist.input} file, the
grid\_rotate program should be run with two command-line options specifying the
original grid file name and the name of the rotated grid file to be produced,
e.g.,

\vspace{12pt}
{\tt > grid\_rotate grid.nc grid\_SE\_Asia\_refinement.nc}
\vspace{12pt}

The original grid file will not be altered, and a new, rotated grid file will be
created. The NCL script {\tt mesh.ncl} may be used to plot either of the
original or rotated grid files after suitable setting the name of the grid file
in the script.

\vspace{12pt}
{\em Note: The grid\_rotate program initializes the new, rotated grid file to a copy of the original grid file.
If the original grid file has only read permission (i.e., no write permission), then so will the copy, and
consequently, the grid\_rotate program will fail when attempting to update the fields in the copy.}
   
